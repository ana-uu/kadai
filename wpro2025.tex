\documentclass{jsarticle}

\usepackage[dvipdfmx]{graphicx}
\usepackage{tabularx}
\usepackage{booktabs}
\usepackage[dvipdfmx]{xcolor}
\usepackage{listings}
\usepackage{url}

\title{ランキングサイト仕様書}
\author{25G1029 大橋慶眞}
\date{\today}

\lstset{
    basicstyle=\ttfamily\small,
    frame={tb},
    breaklines=true,
    numbers=left,
    numberstyle=\color{gray},
    stepnumber=1,
    numbersep=5pt,
    tabsize=4,
    showspaces=false,
    xleftmargin=17pt,
    framexleftmargin=17pt,
}

\begin{document}
\maketitle

\section*{GitHubリポジトリURL}
\url{https://github.com/ana-uu/kadai}

\tableofcontents
\newpage

\section{概要}

本仕様書は,一覧表示をベースとした3つのWebアプリケーションの開発仕様を定義するものである.
各アプリケーションは独立したランキング情報管理システムとして動作し,
Express.jsフレームワークを使用してWebサービスとして提供される.

\subsection{開発対象システム}

\begin{enumerate}
\item FANZA動画ランキングサイト - デジタル動画コンテンツの売れ行きランキング管理
\item DLsiteランキングサイト - 同人作品の売れ行きランキング管理
\item メロンブックスランキングサイト - 同人誌の売れ行きランキング管理
\end{enumerate}

\section{第1部:利用者向け仕様}

\subsection{システム概要}

3つの独立したランキングサイトを提供し,各分野の人気コンテンツの情報を
閲覧・登録・編集・削除できるWebサービスである.

\subsection{利用方法}

\subsubsection{アクセス方法}

各サイトは異なるポートで動作しており,ブラウザで以下のURLにアクセスすることで利用できる.
FANZA動画ランキングは\url{http://localhost:8081/fanza},
DLsiteランキングは\url{http://localhost:8082/dlsite},
メロンブックスランキングは\url{http://localhost:8083/melonbooks}でアクセス可能である.

\subsubsection{基本機能}

\paragraph{ランキング一覧表示}
各サイトのトップページでランキング一覧を表形式で確認できる.
表示される情報は以下の通りである:

\begin{table}[h]
\centering
\caption{各サイトの表示項目}
\begin{tabular}{|l|l|}
\hline
サイト & 表示項目 \\
\hline
FANZA & ランク、動画名、女優名 \\
DLsite & ランク、作品名、サークル名 \\
メロンブックス & ランク、書籍名、著者 \\
\hline
\end{tabular}
\end{table}

\paragraph{詳細情報表示}
一覧のタイトル・作品名・書籍名をクリックすると,
詳細情報ページが表示される.
詳細ページでは各サイトの特性に応じた情報を確認できる.
FANZAサイトではランク、動画名、ジャンル、価格、発売日、女優名、動画リンクが表示され,
DLsiteサイトではランク、作品名、サークル名、ジャンル、価格、評価、作品リンクが表示され,
メロンブックスサイトではランク、書籍名、著者、サークル、ジャンル、価格、書籍リンクが表示される.

\paragraph{新規データ登録}
「新規登録」リンクから新しいランキングデータを追加できる.
必要な情報を入力フォームに記入し,「登録」ボタンを押すことで登録できる.

\paragraph{データ編集・削除}
各データの詳細ページや一覧ページから「編集」リンクをクリックすることで,
既存データの編集が可能である.また,「削除」ボタンから不要なデータを削除できる.
削除時には確認ダイアログが表示され,誤操作を防ぐ仕組みが実装されている.

\subsection{利用上の注意事項}

各サイトは独立して動作するため,データは共有されないことに注意が必要である.
また,登録したデータはサーバー再起動時に初期化されるため,永続的なデータ保存は行われない.

\paragraph{入力値検証}
以下の項目について入力値検証が実装されている:
\begin{itemize}
\item ランク:1以上の整数値のみ許可
\item 価格:0以上の整数値のみ許可
\item 評価(DLsiteのみ):1から5の範囲の数値のみ許可
\item 必須項目:タイトル・作品名・書籍名,リンク,主要項目が空欄でないこと
\item URL形式:リンク項目は有効なURL形式であること
\end{itemize}
不正な値を入力した場合はエラーメッセージが表示され,登録処理は実行されない.

\section{第2部:管理者向け仕様}

\subsection{システム管理}

\subsubsection{サーバー起動・停止}

各サイトは独立したNode.jsプロセスとして動作する.

\paragraph{起動方法}
\begin{lstlisting}
# npm scriptsを使用した起動(推奨)
npm run start-fanza
npm run start-dlsite
npm run start-melonbooks

# 直接起動
node fanza_ranking.js
node dlsite_ranking.js
node melonbooks_ranking.js
\end{lstlisting}

\paragraph{停止方法}
各プロセスでCtrl+Cを押下してサーバーを停止する.

\subsubsection{ポート管理}

各サイトは以下のポートで動作する:

\begin{table}[h]
\centering
\caption{ポート割り当て}
\begin{tabular}{|l|c|}
\hline
サイト & ポート番号 \\
\hline
FANZAランキング & 8081 \\
DLsiteランキング & 8082 \\
メロンブックスランキング & 8083 \\
\hline
\end{tabular}
\end{table}

\subsubsection{データ管理}

現在の実装では,データはメモリ内の配列として管理されている.
サーバー再起動時にはサンプルデータに戻る.

\paragraph{初期データ}
各サイトには5件のサンプルデータが設定されている.

\paragraph{データバックアップ}
本格運用時には,データベースへの移行を推奨する.

\subsection{運用監視}

\subsubsection{ログ監視}
各サーバーの起動時にコンソールに起動メッセージが表示される.
エラー発生時はコンソールログを確認する必要がある.

\subsubsection{アクセス監視}
Express.jsの標準ログ機能により,アクセス状況を監視できる.

\section{第3部:開発者向け仕様}

\subsection{技術仕様}

\subsubsection{開発環境}

本システムの開発にはNode.js(推奨バージョンは14.x以上)、Express.js 4.x、およびEJSテンプレートエンジンが必要である.

\paragraph{環境セットアップ}
依存関係のインストールには以下のコマンドを実行する:
\begin{lstlisting}
npm install
\end{lstlisting}

\paragraph{package.json}
プロジェクトの設定と依存関係を定義するファイル.以下の内容が含まれる:
\begin{itemize}
\item Express.js 4.x - Webアプリケーションフレームワーク
\item EJS 3.x - サーバーサイドテンプレートエンジン  
\item 起動スクリプト - 各サイトの個別起動コマンド定義
\end{itemize}

\paragraph{package-lock.json}
依存関係の正確なバージョンを記録し,異なる環境での再現性を保証するファイル.
npm installの実行時に自動生成される.

\subsubsection{ディレクトリ構成}

\begin{lstlisting}
kadai/
├── fanza_ranking.js      # FANZAサイトメイン
├── dlsite_ranking.js     # DLsiteサイトメイン  
├── melonbooks_ranking.js # メロンブックスサイトメイン
├── package.json          # 依存関係定義
├── package-lock.json     # 依存関係バージョン固定
├── views/                # テンプレートファイル
│   ├── fanza_list.ejs
│   ├── fanza_detail.ejs
│   ├── fanza_new.ejs
│   ├── fanza_edit.ejs
│   ├── dlsite_list.ejs
│   ├── dlsite_detail.ejs
│   ├── dlsite_new.ejs
│   ├── dlsite_edit.ejs
│   ├── melonbooks_list.ejs
│   ├── melonbooks_detail.ejs
│   ├── melonbooks_new.ejs
│   └── melonbooks_edit.ejs
└── public/               # 静的ファイル(未使用)
\end{lstlisting}

\subsection{データ構造}

\subsubsection{FANZA動画ランキングデータ}

\begin{lstlisting}
{
  id: Number,           // 一意識別子(現在の最大ID + 1で自動生成)
  rank: Number,         // ランキング順位
  title: String,        // 動画タイトル
  genre: String,        // ジャンル
  price: Number,        // 価格(円、0以上)
  releaseDate: String,  // 発売日(YYYY-MM-DD)
  actress: String,      // 女優名
  link: String          // 動画リンク(URL)
}
\end{lstlisting}

\subsubsection{DLsiteランキングデータ}

\begin{lstlisting}
{
  id: Number,           // 一意識別子(現在の最大ID + 1で自動生成)
  rank: Number,         // ランキング順位
  productName: String,  // 作品名
  genre: String,        // ジャンル
  price: Number,        // 価格(円、0以上)
  rating: Number,       // 評価(1-5の範囲)
  circle: String,       // サークル名
  link: String          // 作品リンク(URL)
}
\end{lstlisting}

\subsubsection{メロンブックスランキングデータ}

\begin{lstlisting}
{
  id: Number,           // 一意識別子(現在の最大ID + 1で自動生成)
  rank: Number,         // ランキング順位
  bookName: String,     // 書籍名
  author: String,       // 著者
  circle: String,       // サークル
  genre: String,        // ジャンル
  price: Number,        // 価格(円、0以上)
  link: String          // 書籍リンク(URL)
}
\end{lstlisting}

\subsection{API仕様}

\subsubsection{共通エンドポイント}

各サイトで統一されたURL構造を使用する:
\{site\}はFANZAの場合は「fanza」,DLsiteの場合は「dlsite」,メロンブックスの場合は「melonbooks」となる.

\begin{table}[h]
\centering
\caption{APIエンドポイント}
\begin{tabular}{|l|l|l|}
\hline
メソッド & URL & 機能 \\
\hline
GET & /{site} & 一覧表示 \\
GET & /{site}/create & 新規登録フォーム表示 \\
POST & /{site}/create & 新規データ登録処理 \\
GET & /{site}/:id & 詳細情報表示 \\
GET & /{site}/edit/:id & 編集フォーム表示 \\
POST & /{site}/edit/:id & データ編集処理 \\
POST & /{site}/delete/:id & データ削除処理 \\
\hline
\end{tabular}
\end{table}

\subsubsection{レスポンス形式}

\paragraph{HTML レスポンス}
全てのエンドポイントはHTMLページを返す.
EJSテンプレートエンジンを使用してサーバーサイドレンダリングを行う.

\paragraph{リダイレクト}
POST処理後は一覧ページ(/)にリダイレクトする.

\subsection{セキュリティ考慮事項}

本システムのセキュリティ対策として,入力値検証の実装、SQLインジェクション対策、XSS攻撃対策、CSRF攻撃対策などを実装する必要がある。

\end{document}

\subsection{セキュリティ考慮事項}

本システムのセキュリティ対策として,入力値検証の実装、SQLインジェクション対策、XSS攻撃対策、CSRF攻撃対策などを実装する必要がある.

\end{document}